\documentclass{thesis}

\usepackage{tikz}
\usepackage[linguistics]{forest}
\usepackage{rotating}
\usepackage{float}
\usepackage[textsize=tiny]{todonotes}
\usepackage{pgfgantt}
\usepackage[colorlinks=true]{hyperref}
\usepackage{wrapfig}
\usepackage{listings}
\usepackage{flafter}
\usepackage{enumitem}
\setlist{nosep}
\usepackage{sourcecodepro}
\usepackage{lmodern}
\usepackage[T1]{fontenc}
\setlength{\parindent}{0em}
\setlength{\parskip}{1em}
\renewcommand{\baselinestretch}{1.1}
\addtokomafont{disposition}{\rmfamily}

\usepackage{caption}

\DeclareCaptionFormat{myformat}{\hrulefill\\#1#2#3}
\DeclareCaptionLabelFormat{bf-parens}{\textbf{#1~#2}}
\captionsetup[figure]{labelformat=bf-parens,format=myformat}

\usepackage{framed}
\definecolor{shadecolor}{RGB}{216,229,229}

\lstset{basicstyle=\ttfamily,breaklines=true,frame=tb,emph={STREAM}, emphstyle=\underbar,keywordstyle=\bfseries}


\graphicspath{{./fig/}}

\title{Data Generator for Fast Data Benchmarking}

\author{Milinda Pathirage}

\begin{document}

\maketitle

%\tableofcontents

\begin{abstract}
Existing data generators like PDGF and Myriad works well for benchmarking workloads related to Big Data where we process data at rest. Fast data workloads, that process data while they are on their way to become data at rest are inherently different from Big Data workloads. Fast data processing often involve windowed aggregations and joins as well as out-of-order or late arruval handling. Data generators for fast data benchmarks should be able to simulate the real world scenarios that involves late and out-of-order arrivals. There aren't any existing solution that allows generation of high volume, high velocity data for fast data benchmarking. This document discuss such a data generator's design and requirements.
\end{abstract}

\section{Requirements}

\begin{itemize}
	\item It should be possible to simulate a real streaming data workload with following properties:
	\begin{itemize}
	 	\item Millions of distributed actors
	 	\item Out-of-order arrivals
	 	\item Late arrivals
	 	\item Data stream bursts and other abnormalities common in data streams
	 \end{itemize} 
	\item Ability to generate relations that encoded as streams
	\item Support for relation updates
	\item Ability to support custom schema specifications
	\item Ability to extend data output format (Avro, Thrift, JSON) as well as output medium (Kafka, Kinesis)
	\item Able run on top of a modern resource manager such as YARN or Mesos.
\end{itemize}

\section{Design}

\emph{According to PDGF and Myriad papers, considerable amount of effort is needed to implements a distributed data generator. Given that PDGF can already takes care of generating schema's with complex relationships and updates to data, we should wrap PDGF. We can use PDGF to generate data and then add the time dimension and streaming specific features at the wrapper.}

\subsection{Web Call Tree Event Generator}

\begin{itemize}
    \item We have several web pages
    \item Request to one of the above pages generate requests to multiple back-end services.
    \begin{itemize}
        \item Back-end services that get invoked by front-end services may call other web services to gather required data.
    \end{itemize}
    \item Depending on the request to the front-end and depending on the web session (e.g. logged in user or guest user) back-end services invoked by web services may change
    \item Assumes different levels of back-end services and upstream services invoke downstream services not vice versa.
\end{itemize}

\subsubsection{Implementaiton}

\begin{itemize}
    \item Generate random set of web pages
    \item Generate multiple levels of back-end services
    \item Generate multiple call tree templates for each service except services in the lowest level
    \item Above should happen at the master and master should replicate those information to all the slaves which generates the real events
    \item Each generator instance can act as a one partition catring set of users
    \item Number of active users within a given partition should change with time
    \item Need to delay some events for some call trees
\end{itemize}

\begin{figure}[ht!]
{
\scriptsize
\begin{forest}
    for tree={
        grow=0,reversed, % tree direction
        parent anchor=east,child anchor=west, % edge anchors
        edge={line cap=round},outer sep=+1pt, % edge/node connection
        rounded corners,minimum width=15mm,minimum height=8mm, % node shape
        l sep=5mm % level distance
    }
  [WEB\_PAGE\_1
    [SVC\_L1\_A
     [SVC\_L2\_X
       [SVC\_L3\_D]
     ]
     [SVC\_L2\_P
       [SVC\_L3\_B
         [SVC\_L4\_A]
       ]
     ]
    ]
    [SVC\_L1\_C
      [SVC\_L2\_A
        [SVC\_L3\_E]
      ]
      [SVC\_L2\_P
        [SVC\_L3\_B
          [SVC\_L4\_A]
          [SVC\_L4\_R
            [SVC\_L5\_A]
            [SVC\_L5\_C]
          ]
        ]
      ]
    ]
  ]
\end{forest}
}
\caption{Example call tree}
\centering
\end{figure}

\section{TODO}

\begin{itemize}
	\item Use PDGF to generate BigBench test data sample
	\item Understand the BigBench schema and write a tool to feed BigBench schema to Kafka.
\end{itemize}

\section{Notes}

\subsection{08/30/2016}

\begin{itemize}
    \item Decided to go for scenario specific data generator implementations instead of a generic framework like PDGF.
\end{itemize}


\bibliographystyle{abbrv}
\bibliography{bibliography}

\end{document}